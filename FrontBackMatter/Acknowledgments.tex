% % Acknowledgements

% \pdfbookmark[1]{Acknowledgements}{Acknowledgements} % Bookmark name visible in a PDF viewer

% \begin{flushright}{\slshape    
% We have seen that computer programming is an art, \\ 
% because it applies accumulated knowledge to the world, \\ 
% because it requires skill and ingenuity, and especially \\
% because it produces objects of beauty.} \\ \medskip
% --- \defcitealias{knuth:1974}{Donald E. Knuth}\citetalias{knuth:1974} \citep{knuth:1974}
% \end{flushright}

% \bigskip

% %----------------------------------------------------------------------------------------

% \begingroup

% \let\clearpage\relax
% \let\cleardoublepage\relax
% \let\cleardoublepage\relax

% \chapter*{Acknowledgements}

% \noindent I would like to start this thesis by thanking Prof Thorsten Hoeffler, for giving me the opportunity to write my bachelor thesis in his group.

% \noindent A big thank you also goes to Tobias Grosser and Fabian Schuiki, for their continued support and always useful suggestions.

% \noindent Last but not least important, I'd like to thank Martin Erhart, for the big care and thought he always put while working on our MLIR dialect implementation.

% \noindent Put your acknowledgements here.\\

% \noindent Many thanks to everybody who already sent me a postcard!\\

% \noindent Regarding the typography and other help, many thanks go to Marco Kuhlmann, Philipp Lehman, Lothar Schlesier, Jim Young, Lorenzo Pantieri and Enrico Gregorio\footnote{Members of GuIT (Gruppo Italiano Utilizzatori di \TeX\ e \LaTeX )}, J\"org Sommer, Joachim K\"ostler, Daniel Gottschlag, Denis Aydin, Paride Legovini, Steffen Prochnow, Nicolas Repp, Hinrich Harms, Roland Winkler, and the whole \LaTeX-community for support, ideas and some great software.

% \bigskip

% \noindent\emph{Regarding \mLyX}: The \mLyX\ port was initially done by
% \emph{Nicholas Mariette} in March 2009 and continued by
% \emph{Ivo Pletikosi\'c} in 2011. Thank you very much for your work and the contributions to the original style.

% \endgroup