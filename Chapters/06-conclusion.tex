%06-conclusion.tex

\chapter{Conclusion}
With this thesis, we showed that LLHD is fully representable in MLIR (and that our representation is correct) by defining an LLHD dialect and correctly simulating various real-world designs represented in it.
Furthermore, our first implementation of the MLIR-based simulator already comes close to competitive performance. In particular, our simulator manages to surpass a well-established commercial simulator for some test cases, running up to $1.4$ times faster and otherwise only $2.4$ times slower on average. This while being a young and foundational work focused on correctness, with only a few optimizations targeted at performance.
We also showed that the performance losses in our implementation mainly reside in the runtime library and simulator engine, and not in the execution of the lowered LLHD code, making the simulator a good optimization target with high speed-up potential. The small set of optimization we could implement already shows the massive performance speed-up potential present in our simulator. We are confident that further exploration of LLHD-based simulation could enable a new and open-source standard in HDL simulation and verification.

%---------------------------------------------------------------------------------------------------

\section{Future Work}
We think our first implementation opens up many interesting research questions that could be explored in the future, including:

\begin{itemize}
  \item Though initially planned for this thesis, we did not manage to explore what speed-ups we could obtain through a multi-threaded simulation approach. Considering that a significant amount of time in our simulation is spent in creating events, we think adopting a simple \textit{Centralized-queue} algorithm~\cite{Ashenden1994} could already bring significant performance improvements, by splitting that phase over multiple threads.
  \item We did not explore what optimizations and performance improvements other MLIR dialects and transformations could enable, but rather directly lower LLHD to the LLVM dialect, only optimizing the generated LLVM IR code through the provided “O“ flags. Especially with the rise of the CIRCT project, and other hardware-targeted dialects, it will pose an interesting question of what performance improvements they could bring to LLHD simulation.
  \item Introducing static analysis of the input design could potentially bring significant performance improvements. For example, the input design could be analyzed to preemptively allocate the memory buffers necessary to store the various signal updates in the event queue, reducing the costs of dynamic memory allocation. Further analysis could potentially completely avoid the execution of parts of the input design that have no visible effects on the execution. \Eg, consider two processes $A$ and $B$. If $B$'s signal updates always obscure $A$'s updates, then $A$ does not affect the overall behavior of the design, and its execution can be avoided.
  \item Both the LLHD dialect and the LLVM dialect mapping are not complete. \Eg, introducing support for logic types would enable us to generate $100\%$ matching traces when compared against commercial simulators.
        % \item We provided only a limited comparison of our performance results, only comparing it against one commercial simulator. This choice was mainly due to the missing availability of other commercial tools on our testing machine and time constraints. More complete benchmarking and comparison of the current and future implementations of our simulator against a broader range of commercial simulators could give us a clearer picture of the potential of this HDL simulation approach.
\end{itemize}



