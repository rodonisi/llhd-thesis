% 05-related.tex
\chapter{Related Work}
Various open-source HDL simulators exist on the market, but, to our knowledge, none has support for the SystemVerilog standard as vast as LLHD, or does not support SystemVerilog or VHDL at all. Here we list some of the most popular and well-documented open-source alternatives available.

%-------------------------------------------------------------------------------

\section{Verilator}
\textit{Verilator} \cite{verilator} is a popular open-source simulator. It profits from high simulation speeds by compiling SystemVerilog designs into single- or multi-threaded \textit{C++} code, though, mainly synthesizable constructs are accepted. This means that SystemVerilog test-benches cannot be simulated using Verilator as they generally contain unsynthesizable constructs. Instead, the user is expected to write a \textit{C++/SystemC} wrapper around the design under test, (re-)implementing the test-bench. An important distinction that sets Verilator apart from our approach is that Verilator does not have any support for explicit delays, often throwing an error if one is encountered, or otherwise ignoring it.

%-------------------------------------------------------------------------------

\section{SystemC and Cascade}
\textit{SystemC} \cite{systemc} and \textit{Cascade} \cite{Grossman2013}  both define a \textit{C++} library used to model hardware designs. The \textit{C++} constructs can then be used as simulation interfaces, and usually be compiled into an executable performing the simulation.

These approaches, though, do not interface with SystemVerilog or VHDL, but rather expect the designs to be directly implemented through the \textit{C++} constructs. For example, SystemC is often considered as a standalone HDL (although it actually is \textit{C++}) and even takes a spot in Wikipedia's list of HDLs\footnote{\url{https://en.wikipedia.org/wiki/Hardware_description_language\#Examples_of_HDLs}}.

Cascade also takes a cycle-based simulation approach, in contrast to both our approach and SystemC's. This means that simulation state updates are no longer processed on an event-basis, where the time granularity is defined by the events themselves, but rather on fixed clock cycles. Furthermore, Cascade offers Verilog co-simulation, but this is mainly to support Verilog test-benches, rather than entire Verilog designs.

%-------------------------------------------------------------------------------

\section{Icarus Verilog and OSS CVC}
\textit{Icarus Verilog} \cite{icarus} and the \textit{OSS CVC} Verilog compiler \cite{Meyer2016}, to our knowledge, are the two open-source simulation toolchains that most closely resemble our simulation approach.

Icarus Verilog is an open-source implementation of Verilog, mainly focusing on simulation, but also providing support for synthesis. Different from LLHD, though, it focuses on the IEEE Std 1364-2005 Verilog standard \cite{V2006}, with only limited support for the SystemVerilog standard. Most notably, it does not accept any of the examples from our test set. It is otherwise very similar to LLHD's workflow: the input design is first compiled to an intermediate representation, called \textit{vpp}, which can be used for either simulation (through the \textit{vpp} tool) or synthesis.

OSS CVC, on the other hand, compiles Verilog designs directly to x86 machine code and then uses an event-driven approach to simulate the input design. As for Icarus Verilog though, it is only limited to the IEEE Std 1364-2005 Verilog standard, precluding the use of the higher-level SystemVerilog extensions.
